%% Generated by Sphinx.
\def\sphinxdocclass{report}
\documentclass[letterpaper,10pt,english,openany,oneside]{sphinxmanual}
\ifdefined\pdfpxdimen
   \let\sphinxpxdimen\pdfpxdimen\else\newdimen\sphinxpxdimen
\fi \sphinxpxdimen=.75bp\relax

\PassOptionsToPackage{warn}{textcomp}
\usepackage[utf8]{inputenc}
\ifdefined\DeclareUnicodeCharacter
% support both utf8 and utf8x syntaxes
  \ifdefined\DeclareUnicodeCharacterAsOptional
    \def\sphinxDUC#1{\DeclareUnicodeCharacter{"#1}}
  \else
    \let\sphinxDUC\DeclareUnicodeCharacter
  \fi
  \sphinxDUC{00A0}{\nobreakspace}
  \sphinxDUC{2500}{\sphinxunichar{2500}}
  \sphinxDUC{2502}{\sphinxunichar{2502}}
  \sphinxDUC{2514}{\sphinxunichar{2514}}
  \sphinxDUC{251C}{\sphinxunichar{251C}}
  \sphinxDUC{2572}{\textbackslash}
\fi
\usepackage{cmap}
\usepackage[T1]{fontenc}
\usepackage{amsmath,amssymb,amstext}
\usepackage{babel}



\usepackage{times}
\expandafter\ifx\csname T@LGR\endcsname\relax
\else
% LGR was declared as font encoding
  \substitutefont{LGR}{\rmdefault}{cmr}
  \substitutefont{LGR}{\sfdefault}{cmss}
  \substitutefont{LGR}{\ttdefault}{cmtt}
\fi
\expandafter\ifx\csname T@X2\endcsname\relax
  \expandafter\ifx\csname T@T2A\endcsname\relax
  \else
  % T2A was declared as font encoding
    \substitutefont{T2A}{\rmdefault}{cmr}
    \substitutefont{T2A}{\sfdefault}{cmss}
    \substitutefont{T2A}{\ttdefault}{cmtt}
  \fi
\else
% X2 was declared as font encoding
  \substitutefont{X2}{\rmdefault}{cmr}
  \substitutefont{X2}{\sfdefault}{cmss}
  \substitutefont{X2}{\ttdefault}{cmtt}
\fi


\usepackage[Bjarne]{fncychap}
\usepackage{sphinx}

\fvset{fontsize=\small}
\usepackage{geometry}


% Include hyperref last.
\usepackage{hyperref}
% Fix anchor placement for figures with captions.
\usepackage{hypcap}% it must be loaded after hyperref.
% Set up styles of URL: it should be placed after hyperref.
\urlstyle{same}
\addto\captionsenglish{\renewcommand{\contentsname}{Contents:}}

\usepackage{sphinxmessages}
\setcounter{tocdepth}{1}



\title{Nova \sphinxhyphen{} Start your Engines!}
\date{Dec 27, 2020}
\release{}
\author{Vincent Meunier}
\newcommand{\sphinxlogo}{\vbox{}}
\renewcommand{\releasename}{}
\makeindex
\begin{document}

\pagestyle{empty}
\sphinxmaketitle
\pagestyle{plain}
\sphinxtableofcontents
\pagestyle{normal}
\phantomsection\label{\detokenize{index::doc}}



\chapter{Welcome to \sphinxstyleemphasis{Start your Engines!}}
\label{\detokenize{introduction:welcome-to-start-your-engines}}\label{\detokenize{introduction:introduction}}\label{\detokenize{introduction::doc}}
This Nova award is designed to help you explore how technology affects your life each day.

\begin{figure}[htbp]
\centering
\capstart

\noindent\sphinxincludegraphics[width=400\sphinxpxdimen]{{731.imgcache-1024x634}.jpg}
\caption{This jaw\sphinxhyphen{}dropper is the Wärtsilä RT\sphinxhyphen{}flex96C, the world’s largest and most powerful diesel engine. Image obtained from \sphinxhref{https://www.zmescience.com/science/biggest-most-poweful-engine-world/}{zmescience.com}. Check out the website to find out where the engine is used!}\label{\detokenize{introduction:id1}}\end{figure}

\begin{sphinxadmonition}{warning}{Warning:}
When completing this Award both the youth and involved adult leaders must obey all rules of \sphinxhref{https://www.scouting.org/health-and-safety/gss/}{Safe Scouting}. This includes (1) Completing Cyber Chip training prior to starting this activity and (2) \sphinxstylestrong{ALWAYS} involve at least 2 adults in all your communications with a leader, including online. If you send an email to your counselor, always add the address of another adult leader or a parent/guardian. Never reply to a message sent by an adult leader unless another adult has been copied on the email. Report any issue to your parents/guardians!
\end{sphinxadmonition}


\section{Instructions}
\label{\detokenize{introduction:instructions}}\begin{enumerate}
\sphinxsetlistlabels{\arabic}{enumi}{enumii}{}{.}%
\item {} 
Identify a \sphinxstylestrong{Nova Counselor} either within your unit, district, or council.

\item {} 
This site provides you a platform for learning and you can easily follow all requirements using the navigation menu on the left.

\item {} 
Once you have identified a Counselor, you can start working on requirements.

\item {} 
The most important aspect in any scientific endeavor is to \sphinxstylestrong{properly document progress}. This will be done, here, using a google sheet as described in more details below.

\end{enumerate}


\section{Documenting your progress}
\label{\detokenize{introduction:documenting-your-progress}}\begin{enumerate}
\sphinxsetlistlabels{\arabic}{enumi}{enumii}{}{.}%
\item {} 
A template worksheet can be found \sphinxhref{https://docs.google.com/document/d/1Hoqz-rU-vgZ\_VLSfCU9onEyMMCR3jnbiL0DdHXuHA-Y/edit?usp=sharing}{here}. This is a \sphinxstyleemphasis{Google document}. \sphinxstylestrong{You will not be able to modify it until you make your own copy as I will now describe for you.}

\item {} 
Once you have opened the file on google doc, go to \sphinxcode{\sphinxupquote{File}} \(\rightarrow\) \sphinxcode{\sphinxupquote{Make a Copy}}.

\item {} 
Save the file with the following name: \sphinxstyleemphasis{Nova\_designed\_to\_crunch\_FIRSTNAME\_LASTNAME}

\item {} 
You will use that file to enter your progress and share with your counselor.

\item {} \begin{description}
\item[{You can share your own copy of the worksheet with your counselor using the following procedure.}] \leavevmode\begin{enumerate}
\sphinxsetlistlabels{\alph}{enumii}{enumiii}{}{)}%
\item {} 
Click on the SHARE button on the top\sphinxhyphen{}right.

\item {} 
Click on “get link”.

\item {} 
Send the link to your counselor.

\end{enumerate}

\end{description}

\end{enumerate}

\begin{sphinxadmonition}{note}{Note:}
This document provides you a guide to complete the Nova award! All requirements are marked with the following symbol: \(\boxed{\mathbb{REQ}\Large \rightsquigarrow}\). In addition, a number of fun \sphinxstyleemphasis{Additional Challenges} are provided in boxes for your entertainment.
\end{sphinxadmonition}


\section{If you have any question}
\label{\detokenize{introduction:if-you-have-any-question}}
Contact your counselor or your scoutmaster! If you have questions about the program, contact Vincent Meunier  by \sphinxhref{mailto:vinmeunier@gmail.com}{email} (as usual, make sure you copy an additional adult to all your communications with a leader!).


\section{Other Nova modules in this series}
\label{\detokenize{introduction:other-nova-modules-in-this-series}}
\begin{sphinxadmonition}{note}{Science}


\begin{savenotes}\sphinxattablestart
\centering
\begin{tabulary}{\linewidth}[t]{|T|}
\hline

\sphinxhref{https://novashoot.readthedocs.io}{\sphinxincludegraphics[scale=0.65]{{logo-shoot_black}.png}}

\sphinxhref{https://novalig.readthedocs.io}{\sphinxincludegraphics[scale=0.65]{{logo-lig_black}.png}}

\sphinxhref{https://novasplash.readthedocs.io}{\sphinxincludegraphics[scale=0.65]{{logo-splash_black}.png}}

\sphinxhref{https://novamendel.readthedocs.io}{\sphinxincludegraphics[scale=0.65]{{logo-minions_B}.png}}
\\
\hline
\end{tabulary}
\par
\sphinxattableend\end{savenotes}
\end{sphinxadmonition}

\begin{sphinxadmonition}{note}{Technology}


\begin{savenotes}\sphinxattablestart
\centering
\begin{tabulary}{\linewidth}[t]{|T|}
\hline

\sphinxhref{https://novaengines.readthedocs.io}{\sphinxincludegraphics[scale=0.65]{{logo-engines_B}.png}}

\sphinxhref{https://novaworld.readthedocs.io}{\sphinxincludegraphics[scale=0.65]{{logo-world_B}.png}}
\\
\hline
\end{tabulary}
\par
\sphinxattableend\end{savenotes}
\end{sphinxadmonition}

\begin{sphinxadmonition}{note}{Engineering}


\begin{savenotes}\sphinxattablestart
\centering
\begin{tabulary}{\linewidth}[t]{|T|}
\hline

\sphinxhref{https://novawhoosh.readthedocs.io}{\sphinxincludegraphics[scale=0.65]{{logo-whoosh_B}.png}}

\sphinxhref{https://novaupandaway.readthedocs.io}{\sphinxincludegraphics[scale=0.65]{{logo-upandaway_B}.png}}

\sphinxhref{https://novanext.readthedocs.io}{\sphinxincludegraphics[scale=0.65]{{logo-next_B}.png}}
\\
\hline
\end{tabulary}
\par
\sphinxattableend\end{savenotes}
\end{sphinxadmonition}

\begin{sphinxadmonition}{note}{Math}


\begin{savenotes}\sphinxattablestart
\centering
\begin{tabulary}{\linewidth}[t]{|T|}
\hline

\sphinxhref{https://novadtc.readthedocs.io}{\sphinxincludegraphics[scale=0.65]{{logo-dtc2_black}.png}}
\\
\hline
\end{tabulary}
\par
\sphinxattableend\end{savenotes}
\end{sphinxadmonition}


\chapter{Requirement \#1: Research and Reading}
\label{\detokenize{requirement1:requirement-1-research-and-reading}}\label{\detokenize{requirement1::doc}}
\(\boxed{\mathbb{REQ}\Large \rightsquigarrow}\) Choose A or B or C and complete ALL the requirements.
\begin{enumerate}
\sphinxsetlistlabels{\Alph}{enumi}{enumii}{}{.}%
\item {} 
Watch about three hours total of technology\sphinxhyphen{}related shows or documentaries that involve transportation or transportation technology. Then do the following:
\begin{enumerate}
\sphinxsetlistlabels{\arabic}{enumii}{enumiii}{(}{)}%
\item {} 
Make a list of at least two questions or ideas from the show(s) you watched.

\item {} 
Discuss two of the questions or ideas with your counselor.

\end{enumerate}

\begin{sphinxadmonition}{tip}{Tip:}
Some examples include \sphinxhyphen{} but are not limited to \sphinxhyphen{} shows found on PBS (“NOVA”), Discovery Channel, Science Channel. National Geographic Channel, TED Talks (online videos), and the History Channel. You may choose to watch a live performance or movie at a planetarium or science museum instead of watching a media production. You may watch online productions with your counselor’s approval and under your parent’s or guardian’s supervision.
\end{sphinxadmonition}

\item {} 
Read (about three hours total) about transportation or transportation technology. Then do the following:
\begin{enumerate}
\sphinxsetlistlabels{\arabic}{enumii}{enumiii}{(}{)}%
\item {} 
Make a list of at least two questions or ideas from each article.

\item {} 
Discuss two of the questions or ideas with your counselor.

\end{enumerate}

\begin{sphinxadmonition}{tip}{Tip:}
Examples of magazines include \sphinxhyphen{} but are not limited to \sphinxhyphen{} Odyssey, Popular Mechanics, Popular Science, Science Illustrated, Discover, Air \& Space, Popular Astronomy, Astronomy, Science News, Sky \& Telescope, Natural History, Robot, Servo, Nuts and Volts, and Scientific American.
\end{sphinxadmonition}

\item {} 
Do a combination of reading and watching (about three hours total). Then do the following:
\begin{enumerate}
\sphinxsetlistlabels{\arabic}{enumii}{enumiii}{(}{)}%
\item {} 
Make a list of at least two questions or ideas from each article or show.

\item {} 
Discuss two of the questions or ideas with your counselor.

\end{enumerate}

\end{enumerate}

\begin{sphinxadmonition}{note}{Note:}
New transportation technology is invented all the time as humans are constantly driven to explore further and further, pushing the boundaries of what they reach!

\begin{figure}[H]
\centering
\capstart

\noindent\sphinxincludegraphics[width=400\sphinxpxdimen]{{FlyingTaxi}.png}
\caption{Example of futuristic transport technology: the flying taxi. Image obtained from \sphinxhref{https://www.geospatialworld.net/blogs/5-futuristic-transportation-technologies/}{NewScientist GeoSpatialWorld}. Check out the website for a description of a list of fascinating modern and future transport technology.}\label{\detokenize{requirement1:id1}}\end{figure}
\end{sphinxadmonition}

\begin{sphinxadmonition}{attention}{Attention:}
Once you have completed this requirement, make sure you document it in your worksheet!
\end{sphinxadmonition}


\chapter{Requirement \#2: Merit Badge}
\label{\detokenize{requirement2:requirement-2-merit-badge}}\label{\detokenize{requirement2:req2-mb}}\label{\detokenize{requirement2::doc}}
\(\boxed{\mathbb{REQ}\Large \rightsquigarrow}\) Complete ONE merit badge from the following list. Choose one that you have not already used toward another Nova award. After completion, discuss with your counselor how the merit badge you earned uses technology.
\begin{itemize}
\item {} 
Automotive Maintenance

\item {} 
Aviation

\item {} 
Canoeing

\item {} 
Cycling

\item {} 
Drafting

\item {} 
Electricity

\item {} 
Energy

\item {} 
Farm Mechanics

\item {} 
Kayaking

\item {} 
Motorboating

\item {} 
Nuclear Science

\item {} 
Programming

\item {} 
Railroading

\item {} 
Small\sphinxhyphen{}Boat Sailing

\item {} 
Space Exploration

\item {} 
Truck Transportation

\end{itemize}

\begin{figure}[htbp]
\centering

\noindent\sphinxincludegraphics[width=700\sphinxpxdimen]{{meritbadges}.png}
\end{figure}

\begin{sphinxadmonition}{attention}{Attention:}
Once you have completed this requirement, make sure you document it in your worksheet!
\end{sphinxadmonition}


\chapter{Requirement \#3: Hands\sphinxhyphen{}on activities}
\label{\detokenize{requirement3:requirement-3-hands-on-activities}}\label{\detokenize{requirement3::doc}}
\(\boxed{\mathbb{REQ}\Large \rightsquigarrow}\) Do ALL of the following:
\begin{enumerate}
\sphinxsetlistlabels{\Alph}{enumi}{enumii}{}{.}%
\item {} 
Using the requirements from the list of merit badges provided in {\hyperref[\detokenize{requirement2:req2-mb}]{\sphinxcrossref{\DUrole{std,std-ref}{Requirement \#2}}}}.
\begin{enumerate}
\sphinxsetlistlabels{\arabic}{enumii}{enumiii}{(}{)}%
\item {} 
Tell your counselor the energy source(s) used in these merit badges.

\item {} 
Discuss the pros and cons of each energy source with your counselor.

\end{enumerate}

\item {} 
Make a list of sources of energy that may be possible to use in transportation.

\item {} 
With your counselor:
\begin{enumerate}
\sphinxsetlistlabels{\arabic}{enumii}{enumiii}{(}{)}%
\item {} 
Discuss alternative sources of energy.

\item {} 
Discuss the pros and cons of using alternative energy sources.

\end{enumerate}

\end{enumerate}

\begin{sphinxadmonition}{note}{Note:}
\sphinxstylestrong{What is renewable energy?}

Renewable energy is made from resources that nature will replace, like wind, water and sunshine. Renewable energy is also called “clean energy” or “green power” because it doesn’t pollute the air or the water.

One reason we can’t use renewable energy for everything (yet!) is that we can’t store up wind or sunshine, the same way we store gas in a tank or energy in coal. In addition, even though costs have decreased very quickly, it is still more expensive to generate electricity with solar or wind technology. However, scientists and engineers are working hard! As a result, costs are steadily decresing and storage technology is improving steadily as well.

Learn more about renewable energy \sphinxhref{https://www.ducksters.com/science/environment/renewable\_energy.php}{here} or \sphinxhref{https://c03.apogee.net/mvc/home/hes/land/el?utilityname=pseg\&spc=kids\&id=16183}{here}. The second link has lots of information on energy, check it out!

\begin{figure}[H]
\centering
\capstart

\noindent\sphinxincludegraphics[width=400\sphinxpxdimen]{{renewable}.png}
\caption{Examples of renewable energy technology.  Image obtained from the \sphinxurl{https://c03.apogee.net/mvc/home/hes/land/el?utilityname=pseg\&spc=kids\&id=16183}  website.}\label{\detokenize{requirement3:id1}}\end{figure}
\end{sphinxadmonition}

\begin{sphinxadmonition}{attention}{Attention:}
Once you have completed this requirement, make sure you document it in your worksheet!
\end{sphinxadmonition}


\chapter{Requirement \#4: Building}
\label{\detokenize{requirement4:requirement-4-building}}\label{\detokenize{requirement4::doc}}
\begin{figure}[htbp]
\centering
\capstart

\noindent\sphinxincludegraphics[width=400\sphinxpxdimen]{{CardboardCar}.jpg}
\caption{This is a cardboard car, powered by a human on a bicycle. Not all projects are energy efficient or even useful: they can be fun too! (image from \sphinxhref{http://www.kimimaeda.com/cardboard-car/}{kimimaeda.com})}\label{\detokenize{requirement4:id1}}\end{figure}

\(\boxed{\mathbb{REQ}\Large \rightsquigarrow}\) Design and build a working model vehicle (not from a kit).
\begin{enumerate}
\sphinxsetlistlabels{\Alph}{enumi}{enumii}{}{.}%
\item {} 
Make drawings and specifications of your model vehicle before you begin to build.

\item {} 
Include one of the following energy sources to power your vehicle  (do not use gasoline or other combustible fuel source): solar power, wind power, or battery power.

\item {} 
Test your model. Then answer the following questions:
\begin{enumerate}
\sphinxsetlistlabels{\arabic}{enumii}{enumiii}{(}{)}%
\item {} 
How well did it perform?

\item {} 
Did it move as well as you thought it would?

\item {} 
Did you encounter problems? How can these problems be corrected?

\end{enumerate}

\item {} 
Discuss with your counselor:
\begin{enumerate}
\sphinxsetlistlabels{\arabic}{enumii}{enumiii}{(}{)}%
\item {} 
Any difficulties you encountered in designing and building your model

\item {} 
Why you chose a particular energy source

\item {} 
Whether your model met your specifications

\item {} 
How you would modify your design to make it better

\end{enumerate}

\end{enumerate}

\begin{sphinxadmonition}{tip}{Tip:}
Building your vehicle

There are plenty of sources of information on the internet to meet the requirements for this activity. Please choose a project that is \sphinxstyleemphasis{doable} but sufficiently \sphinxstyleemphasis{challenging}. And remember: having fun is important.

For example, the \sphinxstylestrong{balloon car} is a very nice entry\sphinxhyphen{}level project (you can find information on the Youtube video below:


\end{sphinxadmonition}

\begin{sphinxadmonition}{note}{Additional Challenge}

More advanced vehicles include the \sphinxstyleemphasis{mousetrap cars}. There is also a lot of information on those on the web. However, if you attempt this: watch out for your fingers!!!


\end{sphinxadmonition}

\begin{sphinxadmonition}{attention}{Attention:}
Once you have completed this requirement, make sure you document it in your worksheet!
\end{sphinxadmonition}


\chapter{Requirement \#5: Technology @ life}
\label{\detokenize{requirement5:requirement-5-technology-life}}\label{\detokenize{requirement5::doc}}
\(\boxed{\mathbb{REQ}\Large \rightsquigarrow}\) Discuss with your counselor how technology affects your everyday life.

\begin{sphinxadmonition}{note}{Additional Challenge}

You have certainly heard about the self\sphinxhyphen{}driving car developped by a number of technology companies. With the help of your counselor or a guardian, perform research on the topic. Consider both the software and hardware challenges associated with the development of self\sphinxhyphen{}driving cars. Find out more about safety as well.
\end{sphinxadmonition}

\begin{figure}[htbp]
\centering
\capstart

\noindent\sphinxincludegraphics[width=500\sphinxpxdimen]{{transportation-technology-pillar-page-types-of-transportation_0}.jpg}
\caption{An artist’s view of a future electric train powered by renewable energy. (Image obtained from \sphinxurl{https://builtin.com/transportation-tech}). Check out the website for many other cool examples of modern transport technologies!}\label{\detokenize{requirement5:id1}}\end{figure}

\begin{sphinxadmonition}{attention}{Attention:}
Once you have completed this requirement, make sure you document it in your worksheet!
\end{sphinxadmonition}


\chapter{About the author}
\label{\detokenize{contact:about-the-author}}\label{\detokenize{contact::doc}}
These pages were written by Vincent Meunier, the Chair of the STEM committee of \sphinxhref{https://www.trcscouting.org}{Twin Rivers Council} in New York State.

Vincent Meunier is a Professor of physics at Rensselaer Polytechnic Institute. If you have any questions, feel free to contact him by \sphinxhref{mailto:vinmeunier@gmail.com}{email}.

\begin{sphinxadmonition}{note}{Note:}
Most of the material used here was obtained from a number of external scouting sources, including \sphinxhref{https://www.scouting.org/wp-content/uploads/2018/11/Designed-to-Crunch-Nova-2018Nov26.pdf}{scouting.org}
\end{sphinxadmonition}

\begin{sphinxadmonition}{note}{Note:}
An engine, or motor, is a machine used to change energy into movement that can be used. The energy can be in any form. Common forms of energy used in engines are electricity, chemical (such as petrol or diesel) or heat. When a chemical is used to produce energy it is known as fuel.
\end{sphinxadmonition}

The difference of engine and motor is that an engine creates mechanical energy from heat, while motor creates mechanical energy from other kinds of energy, like electricity. Typical engines are steam engine and internal combustion engine, while typical motors are electric motor and hydraulic motor. (text adapted from \sphinxhref{https://kids.kiddle.co/Engine}{Kiddle}). Check out the link to learn plenty of information on engines!
\begin{quote}

\begin{figure}[htbp]
\centering
\capstart

\noindent\sphinxincludegraphics[width=400\sphinxpxdimen]{{Stirling-Engine-Kit-4-Cylinder_600x}.png}
\caption{Stirling Engine Kit (Image obtained from enginediy.com)}\label{\detokenize{index:id1}}\end{figure}
\end{quote}



\renewcommand{\indexname}{Index}
\printindex
\end{document}